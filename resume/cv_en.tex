\insertcvheader

%%LANGUAGES

\cvplaincontent{
	\textbf{Languages}~: French, English (spoken and written)
}

%% SKILLS

\cvsection{SKILLS}

\cvplaincontent{	
	\textbf{Operating Systems}~: Linux, Windows, Mac OS X\\
	\textbf{Languages}~: C++, GLSL, JavaScript, Python, Java, Latex, SVG\\	
	\textbf{Libraries}~: Qt, OpenGL, CUDA, VTK, NodeJS, Electron, Unity\\	
	\textbf{CG Techniques}~: path tracing, volume rendering, shadow casting
}

%% EDUCATION

\cvsection{EDUCATION}

\cventry{2015~--~2017}{Master of Computer Science}{Polytechnique Montréal}{
	Parallel mesh adaptation on the CPU and GPU\\
	Research presented at USNCCM 14
}

\cventry{2010~--~2015}{Bachelor of Software Engineering}{Polytechnique Montréal}{
	Cumulative mean~: 3,93/4\\
	Multimedia orientation
}


%% WORK

\cvsection{WORK EXPERIENCE}

\cventry{Summer 2014}{Software Engineering Internship}{Dental Wings}{
	Development of a new intraoral scanning station for digital dental impression\\
	Implement a shadow casting engine in OpenGL/GLSL\\
	Design jaw manipulation gestures for the Leap Motion\\
	Web page~: \href{http://www.dentalwings.com/products/intraoral-scanner/}{dentalwings.com/products/intraoral-scanner}\\
	Working with the Agile/Scrum development processes	
}

\cventry{2013~--~2014}{Technical Director for MÉTIS}{Polytechnique Montréal}{
	In partenership with the Centre de réadaptation Marie Enfant du CHU Sainte Justine\\
	The main goal was to design of a motorized wheelchair for young children\\
	Manage the software team within a multidisciplinary student project\\
	Make weekly updates on task progression to the board of directors\\
	Evaluate and select the appropriate technologies for the project\\
	Document the chosen solutions and the installation procedures
}

\cventry{2012~--~2013}{Software Engineering Internship}{Zimmer CAS}{
	Automate the assembly of knee replacement surgical tools\\
	Script SOLIDWORKS geometrical operations in C\# and C++ CLI\\
	Manipulate and render medical MRI with the help of GDCM and VTK\\
	Validate third party libraries and document in-house software\\
	Design and refactor graphical user interfaces with Qt\\
	Enterprise's management practices were shifting from Waterfall to Agile
}

\cventry{Summer 2011}{Restructuration du cours INF1010}{Polytechnique Montréal}{
	Design exercises and examples about object-oriented programming\\
	Write lecture notes on the STL library and graphical interfaces\\
	Refactor the course's website on \textit{Moodle}\\
	Conception de mini-évaluations pour les élèves
}


%% PROJECTS

\cvsection{PROJECTS}

\cventry{2015~--~2016}{Production d'un court métrage d'animation}{Personal Project}{
	\textit{The Fruit -- A Moving Picture}~: \href{http://wibus.github.io/ExTh}{wibus.github.io/ExTh}\\
	Logiciels d'animation, rendu et post-production développés expressément pour cette animation\\
	Le système produit des images par path tracing sur CPU de manière parallèle et distribuée\\
	Le logiciel de post production réduit le bruit, applique des effets d’éblouissements, fait un mappage de ton local temporellement et applique une correction gamma sur les images\\
	La scène ne contient aucun triangle; tous les objets sont le résultat d'opérations booléennes sur des surfaces analytiques quadriques (plans, sphères, paraboloïdes, hyperboloïdes)
}

\cventry{2011~--~2016}{Conception d'un moteur de simulations}{Personal Project}{
	\textit{Experiemental Theatre} est une librairie  C++ qui regroupe les quatre modules suivants en exploitant la métaphore du théâtre pour nommer les classes et définir leurs responsabilités~:\\
	$\bullet$ {Cellar Workbech}~: Structure de données, patrons de conception génériques, journalisation, primitives géométriques, gestion de ressources, classes wrapper pour OpenGL, etc.\\
	$\bullet$ \textit{Prop Room 2D}~: Framework de rendu et de physique newtonienne pour des géométries 2D\\
	$\bullet$ \textit{Prop Room 3D}~: Framework de rendu par tracer de chemin parallèle et distributé sur CPU\\
	$\bullet$ \textit{Scaena}~: Framework pour la création et la gestion d'entités dans une simulation ou un jeu\\
	Une dizaine de projets personnels sont basés sur cette librairie, dont mon projet de maîtrise\\
	Disponible sur Github : \href{http://www.github.com/wibus/ExperimentalTheatre}{github.com/wibus/ExperimentalTheatre}
}

\cventry{Spring 2015}{Développement d'un jeu vidéo pour Laval Virtual}{Integrator Project}{
	Jeu de course à obstacles pour la Kinect avec compagnons sur tablettes électroniques\\
	Les compagnons ennemis ajoutes des obstacles pour freiner le coureur d'avantage\\
	Les compagnons alliés déposent des objets sur le parcours pour avantager le coureur\\
	Les tablettes des compagnons sont connectées par Wifi à l'ordinateur qui exécute jeu\\
	La trajectoire du parcours est générée aléatoirement pour une expérience unique à chaque fois\\
	Le jeu est physiquement exigeant pour le coureur et favorise la coopération entre compagnons
}

\cventry{Spring\\2012 et 2014}{Développement d'un jeu de hockey sur coussin d'air}{Integrator Project}{
	Interfaçage d'une application Java avec un framework C++ par JNI\\
	Conception d'une architecture logicielle basée sur les patrons Composite et Visiteur\\
	Application de patrons de conception~: Façade, Commande, Singleton, Observateur, etc.\\
	Conception de l'interface graphique en Swing\\
	Implémentation de la physique de jeu en 2D\\
	Rendu de la scène 3D à l'aide d'OpenGL
}

\cventry{Spring 2011}{Assemblage et programmation d'un robot}{Integrator Project}{
	Montage de la carte mère pour accueillir un microcontrôleur ATMega16 et un pont en H\\
	Assemblage et installation d'engrenages pour un petit moteur électrique\\
	Interfaçage avec un capteur magnétique, un capteur infrarouge et d'une mémoire externe\\
	Production de mélodies à l'aide du module PWM et d'un haut-parleur piézoélectrique\\
	Développement d'un programme C++ pour la traversée d'un parcours à obstacles
}


%% SCOLARSHIPS

\cvsection{SCOLARSHIPS}
 
\cventry{Winter 2015}{Bourse du CRSNG}{}{
	Bourse d’entrée à la maîtrise recherche
}

\cventry{Winter 2015}{Bourse du FRQNT}{}{
	Bourse d’entrée à la maîtrise recherche
}

\cventry{Winter 2011}{Prix Philip et Lily Malouf}{}{
	Équipe gagnante du projet intégrateur de première année
}

\cventry{Autumn 2010}{Bourse du directeur général}{}{
	Excellence du dossier scolaire au CÉGEP
}

%% HOBIES

\cvsection{ACTIVITIES}

\cvplaincontent{
	\textbf{Arts}~: Guitare, Clarinette, Illustration\\
	\textbf{Sports}~: Nage, Vélo, Unicycle
}

\end{document}