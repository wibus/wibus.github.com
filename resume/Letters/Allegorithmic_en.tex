#Experience Summary :

During my education, I have cumulated a total of 18 months of full-time internship. Most of that time was dedicated to Zimmer CAS. First, I was enhancing the user interface and implementing new features for a knee suregery triage aplication. Second, acknowledging my skills in software development, the Team Lead entrusted me with the design and development of a new Jig Creation application. This software fully automated the assembly of personalized surgical guides based on the patient's anatomical measurements. I also worked at Dental Wings where my team developed a modern rendering engine in OpenGL for the company's next introral scanning stations.

Yet, my professional experience only account for half of my pratical exprience. The other half comes from my countless personal projects. I learned coding with Game Maker at the age of fourteen and never stopped making games, experiments and simulations since then. The most significant project of them all was a parallel and distributed path tracer written in C++. I have spent around 2000 hours learning the basic theory of path tracing, writing the renderer from scratch and designing the animation and composition software to produce a single short film. I do not consider having a complete understanding of the path tracing theory, so I dedicate myself to learning more by reading Physically Based Rendering: From Theory to Implementation.

Most of my jobs and personal projects made me work with Qt and OpenGL 4 in C++, which make up my favorite develoment environment by far. Lately, through my last major project, I have added Electron as well as NodeJS, Javascript, HTML and CSS to my repertoire of well known technologies.


#Cover Letter :

Dear Hiring Manager of software developers,

By reading the job opening's description for your Montreal office, I instantly recognized myself. I have been programming for twelve years now, and found out early that what I was aiming for, as a career, was to engineer tools for visual artists and I feel like Allegorithmic has this exact same aim.

As soon as I had classes about C++ that gave me a much better understanding of the language, I staterted experimenting with OpenGL. I made myself familiar with different classical rendering techniques like volume rendering, radiosity, shadow mapping and image based lighting. At the end of my Bachelor's degree, I learned about path tracing and it soon became my favorite rendering algorithm. Since then, I never stopped experimenting with the algorithm and this is what led me to the production a of short animated film.

For my Master's degree, I wanted to do research on computer graphics but did not find a laboratory at Polytechnique that would fit my needs. So I took my second choice, mesh generation for numerical simulation; a decision I did not regret. Receiving three scholarships at my admission, I had the opportunity to define my research subject from A to Z. I choose to study parallel mesh adaptation on the GPU. To test my ideas, I reused the same C++ framework I had been developing for my personal projects in the last four years. My first GPU implementations were done as OpenGL computer shaders, but the drivers were too unstable. So, alongside these implementations, I rewrote everything in CUDA. It was surprising to see that CUDA was not only more stable but also faster. In the end, I had a geometric adaptation operator that was running 4.5 times faster multi-threaded, 20 times faster in GLSL and 45 times faster in CUDA then its serial implementation. As I intended, this project succeeded in making general-purpose programing on GPU one of my main area of expertise.

As a final word, I will mention how I am as I team member. In my previous jobs, most of the projects I worked on were managed as Scrum projects. I really appreciated the kind of interactions it brough to the team, especially the fact that we had daily meetings. I am confortable in close knit teams and enjoy the sense of flexibility and global awareness that Scrum provides.

I am available for an interview at your convenience. Please do not hesitate to browse my résumé.

Thank you for your consideration,


William Bussière
