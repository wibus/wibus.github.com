\insertcvheader
\clrecipient
 {Christiana Simonsen}
 {Gestionnaire des Ressources Humaines}
 {Maya Heat Transfer Technologies Ltd.}
 {4999 Rue St-Catherine Ouest, Suite 400}
 {Montréal, Québec H3Z 1T3, CANADA}
 
\cllocationdate
 {Montréal}
 {4 septembre}
 
\clgreeting
 {Madame}
 
Je viens tout juste de compléter ma maîtrise à l’École Polytechnique de Montréal et suis actuellement à la recherche d’un emploi en développement de logiciel. Mon directeur de recherche (François Guibault), voyant mon intérêt pour la mécanique des fluides numérique, m’a encouragé à postuler chez Maya HTT sachant que plusieurs de ses anciens étudiants s’y sont joints.

En regardant de plus près, j’ai constaté qu’effectivement le poste de développeur logiciel CAO-IAO représente une excellente opportunité d’appliquer mes connaissances en génération, adaptation et visualisation de maillage. 
 
Bien que mon expérience en entreprise se limite aux 14 mois de stage que j’ai complétés pendant mes études, j’ai développé une grande expertise en infographie et en développement de logiciel par le biais de projets personnels d’envergure. Je me suis d’abord intéressé à la mécanique 2D. C’est ainsi que j’ai conçu un cadre d’application pour l’exécution et la visualisation de simulations à l’aide des librairies Qt et OpenGL. Une fois ce projet terminé, j’ai entrepris la production d’un court film d’animation dont les images sont produites par \textit{path tracing}. J’ai écrit l’intégralité des logiciels impliqués dans cette production~: du logiciel de rendu au logiciel d’animation en passant par le logiciel de postproduction. Finalement, mon projet de maîtrise m’a amené à écrire une application pour visualiser et adapter des maillages volumiques en mécanique des fluides numériques. Mon mémoire de maîtrise porte sur l’accélération du processus d’adaptation de maillage par parallélisation sur processeurs graphiques (GPU). Celui-ci est sélectionné comme meilleur mémoire de l’année 2017 à Polytechnique.

Je travaille également très bien en équipe sur des projets d’entreprise. Mon premier stage s’est déroulé dans une entreprise biomédicale, où la fiabilité des produits délivrés était une priorité. C’est ainsi que j’ai mis à l’épreuve ma capacité à vérifier des composantes, à écrire des tests et corriger des défauts logiciels tout en respectant un échéancier strict. Mon deuxième consistait à développer une nouvelle plateforme de numérisation dentaire intraorale. Un soin particulier était porté à la convivialité de l’interface et aux performances de l’application.

En somme, je crois posséder les compétences techniques et la curiosité nécessaire pour contribuer significativement aux solutions de CAO et IOA chez Maya HTT. Sans compter que travailler dans le domaine de la thermique et de la mécanique des fluides représenterait une source inépuisable de motivation pour moi. 

Je suis disponible pour une entrevue et n’hésitez pas à consulter mon CV ci-inclus.
Veuillez agréer, Madame, mes sincères salutations

\clsignature{William Bussière}