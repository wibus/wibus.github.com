\clurecipient
 {Polytechnique Montréal}
 {2900 boul. Edouard-Monpetit}
 {Montréal, QC, H3T 1J4}
 
\cllocationdate
 {Montréal}
 {14 décembre}
 
Objet~: Précisions sur la durée d'études\\
 
\clgreeting
 {Madame, Monsieur}
 
Mon programme de maîtrise s'est étendu de l'été 2015 à la session d'été 2017, ce qui couvre sept (7) sessions d'études. Ayant obtenu à la fois la bourse du CRSNG et du FRQNT, j'ai décidé d'élaborer mon propre sujet de recherche afin de travailler sur des problèmes plus près de mes champs d'intérêt.

Au cours de ma recherche, j'ai conçu l'intégralité d'une application pour comparer diverses techniques d'adaptation de maillages. Cette application est actuellement publiée sous forme de code source ouvert et accessible depuis le site web \textit{GitHub}. Elle représente un effort considérable de conception et de développement logiciel de ma part.

Toutefois, la raison principale pour laquelle mon projet s'est étendu sur plus de six sessions découle d'une constatation faite à l'automne 2016. À l'époque, j'obtenais des résultats relativement positifs, mais peu satisfaisants en comparaison avec des problèmes similaires. En poussant la réflexion sur l'amélioration de la technique développée, j'ai constaté qu'en parallélisant la résolution selon une nouvelle division du travail, j'arriverais à multiplier les performances de cette technique. J'avais alors deux options~: terminer mon mémoire avec les résultats actuels et discuter de la nouvelle solution en tant que travail futur ou implémenter la nouvelle solution et mettre à jour mon mémoire en fonction de celle-ci. Je savais que la deuxième option représentait plusieurs mois de travail et qu'elle repousserait mon dépôt d'au moins une session. Toutefois, c'est l'option que j'ai choisie et je ne regrette pas mon choix puisque cette nouvelle division du travail constitue ma plus importante contribution à la recherche. De plus, je suis certain que cet ajout a eu un impact significatif sur la décision du jury de recommander mon travail en tant que meilleur mémoire de l'année 2017.

Je vous remercie pour l'attention que vous portez à ma candidature,

\clsignature{William Bussière}