\insertcvheader

\clurecipient
 {Maxon Computer Canada}
 {434 Rue Saint-Pierre \#400}
 {Montréal, QC, H2Y 2M5}
 
\cllocationdate
 {Montréal}
 {8 janvier}
 
\textit{Note: English resume and cover letter available on demand.}\\
 
\clgreeting
 {Madame, Monsieur}
 
Venant tout juste de compléter ma maîtrise à l’École Polytechnique de Montréal, je suis actuellement à la recherche d’un emploi en développement de logiciel de rendu 3D ou 2D. J'ai obtenu un emploi à la sortie de mes études, mais malheureusement le projet auquel j'étais assigné a été annulé avant même la complétion de mon premier mois de travail.

Deux postes ont capté mon attention chez vous. Le premier, et celui pour lequel je semble être le plus qualifié, est le poste de \textit{Junior C++ Developer (Modeling)}. Pour un développeur junior, j'ai une excellente compréhension et expertise technique en modélisation et rendu. Dans mon CV, vous constaterez que mon mémoire de maîtrise portait sur la génération et l'adaptation de maillages volumiques pour la mécanique des fluides numérique. J'ai également acquis de l'expérience au sujet des modèles analytiques et paramétriques au fils de mes nombreux projets personnels et de mes cours aux cycles supérieurs~: \textit{Conception géométrique assistée par ordinateur et visualisation}, \textit{Infographie avancée} et \textit{Mathématiques des éléments finis}. Depuis plus de cinq ans, je suis à l'affût des mise-à-jour du langage C++ et d'OpenGL en incorporant leurs nouvelles fonctionnalités au fil de mes projets. Je regarde aussi activement Vulkan, que j'espère bientôt ajouter à mon expertise. 

Le deuxième poste qui m'intéresse particulièrement est celui de \textit{GPU Rendring Specialist}. Mon objectif de carrière à long terme est de travailler sur un path tracer de production. Vous remarquerez que mon projet personnel le plus important est le développement d'un path tracer unidirectionnel ainsi que d'une suite logicielle pour la visualisation temps réel, l'animation ainsi que la post-production de films. Le rendu est parallélisé sur CPU et distribué sur plusieurs nœuds. Ces suite logicielle m'a permis de produire un court métrage (The Fruit -- A moving Picture). En soit, ce projet m'a permis de gagner énormément d'expérience en rendu offline, mais j'ai décidé d'étudier le rendu basé sur la physique plus sérieusement en faisant de \textit{Physically Based Rendering: From Theory to Implementation} mon livre de chevet.
 
Outre mon expérience technique et ma motivation, je crois que vous serez intéressé d'entendre que je suis un très bon joueur d'équipe. Pratiquement tout mes environnements de travail était gérés de manière Agile et j'ai grandement apprécié le type d'interactions que cela amenait dans les équipes de travail. Personnellement, je me décrirais 

Je suis disponible pour une entrevue et n’hésitez pas à consulter mon CV ci-inclus.

Veuillez agréer, Madame, Monsieur, mes sincères salutations

\clsignature{William Bussière}