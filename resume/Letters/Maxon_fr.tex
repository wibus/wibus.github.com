\insertcvheader

\clrecipient
 {M. Jean-François Yelle}
 {Managing Director}
 {Maxon Computer Canada Inc.}
 {434 Rue Saint-Pierre \#400}
 {Montréal, QC, H2Y 2M5}
 
\cllocationdate
 {Montréal}
 {9 janvier}
 
\textit{Note: English resume and cover letter available on demand.}\\
 
\clgreeting
 {Monsieur}
 
Venant récemment de compléter ma maîtrise à l’École Polytechnique de Montréal, je suis actuellement à la recherche d’un emploi en développement de logiciel de rendu 3D ou 2D. Notez que dès la sortie de mes études, j'ai obtenu un emploi chez Autodesk pour travailler sur le dernier mois d'un projet de visualisation architecturale qui a confirmé ma passion pour l'infographie.

Deux postes ont capté mon attention chez Maxon. Le premier, celui pour lequel je semble être le plus qualifié, est le poste de \textit{Junior C++ Developer (Modeling)}. Malgré mon statut junior, j'ai déjà su développer une excellente expertise technique en modélisation et en rendu graphique. Dans mon CV, vous constaterez que mon mémoire de maîtrise portait sur la génération et l'adaptation de maillages volumiques pour la mécanique des fluides numérique. J'ai également acquis de l'expérience avec les modèles analytiques et paramétriques au fils de mes nombreux projets personnels et de mes cours aux cycles supérieurs~: \textit{Conception géométrique assistée par ordinateur et visualisation}, \textit{Infographie avancée} et \textit{Mathématiques des éléments finis}. Depuis plus de cinq ans, je suis à l'affût des nouveautés du langage C++ et d'OpenGL que j'incorpore au fil de mes projets. Je regarde aussi activement Vulkan, que j'espère bientôt pouvoir ajouter à mon basin d'expertise. 

Le deuxième poste qui m'intéresse particulièrement est celui de \textit{GPU Rendering Specialist}. Mon objectif de carrière à long terme est de travailler sur un path tracer de production. Vous remarquerez que mon projet personnel le plus ambitieux consistait à développer un path tracer unidirectionnel ainsi que d'une suite logicielle pour la visualisation temps réel, l'animation ainsi que la postproduction de films. Le rendu était parallélisé sur CPU et distribué sur plusieurs ordinateurs. Cette suite logicielle m'a permis de produire un court métrage (The Fruit -- A moving Picture). Ce projet m'a également permis de gagner énormément d'expérience en rendu offline sans toutefois m'offrir de base théorique solide. C'est pourquoi j'ai décidé d'étudier le rendu basé sur la physique plus sérieusement en faisant de \textit{Physically Based Rendering: From Theory to Implementation} mon livre de chevet.

 
Outre mon expérience technique et ma motivation, je crois que vous serez intéressé d'entendre que je suis un très bon coéquipier. Pratiquement tous mes environnements de travail étaient gérés de manière agile et j'ai grandement apprécié le type d'interactions que cela amenait dans les équipes de travail. En somme, le travail coopératif me convient tout aussi bien que le travail autonome.

Je suis disponible pour une entrevue et n’hésitez pas à consulter mon CV ci-inclus.

Veuillez agréer, Monsieur, mes sincères salutations

\clsignature{William Bussière}