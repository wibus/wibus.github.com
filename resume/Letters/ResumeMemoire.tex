\section*{\hfil RÉSUMÉ\hfil}

La mécanique des fluides numérique est le domaine qui modélise et analyse les phénomènes liés aux fluides tels que l'air et l'eau. On peut aborder ces phénomènes d'un point de vu technique, par exemple en étudiant la circulation d'air autour d'une ail d'avion. On peut également s'y intéresser d'un point divertissement si on veut augmenter le réalisme des films d'animation ou des jeux vidéos en simulant le courant d'une rivière ou l'évolution des nuages dans le ciel. Toutefois, la simulation de fluides requiert une très grande puissance de calcul, sans compter que plus la précision souhaitée est grande, plus le temps de calcul sera grand. Concrètement, on parle de plusieurs heures pour un problème stationnaire à plusieurs jours pour un problème instationnaire. C'est pourquoi on exécute généralement ces simulations sur des ordinateurs très puissants tel que des stations de travail dédiées ou des super-ordinateurs.

L'adaptation de maillage permet de maximiser la précision d'une simulation tout en minimisant le temps de calcul et la mémoire utilisée. Cette méthode produit des représentations géométriques sur mesure des problèmes étudiés. Cependant, les implémentations actuelles de cette méthode restent très gourmandes en temps pour des problèmes tridimensionnels. Sans compter qu'un effort de parallélisation des méthode de résolution est déjà engagé. Si les méthode d'adaptation ne bénéficient pas du même effort, leur temps de calcul domineront bientôt le temps total d'analyse numérique. La solution proposée dans ce mémoire consiste à paralléliser l'étape d'adaptation de maillage sur des processeurs graphiques (GPU). Puisque qu'un maillage possède généralement quelques millions de sommets et que les processeurs graphiques actuels possèdent plusieurs milliers de coeurs pouvant travailler en parallèle, l'adaptation de maillage est un problème \textit{a priori} bien posé pour obtenir une bonne accélération sur processeur graphique. Effectivement, les résultats démontrent que qu'une implémentation sur processeur graphique est environ 50 fois plus rapide qu'une implémentation séquentielle et 10 fois plus rapide qu'une implémentation parallèle sur l'unité centrale de traitement (CPU).

Pratiquement tous les ordinateurs de nos jours possède au moins une carte graphique. Ce nouveau type de parallélisation est donc accessible à un vaste public, réduisant ainsi la dépendance des utilisateurs aux stations de travail puissantes ainsi qu'aux super-ordinateurs. De plus, en accélérant le processeur d'adaptation de maillage, l'ensemble du temps de simulation est réduit ce qui permet aux ingénieurs de faire plus de tests ou de proposer plus de concepts en un temps donné. Bien que plusieurs recherches aient obtenue des résultats similaires en optimisation de maillage, la présente recherche est la première à s'intéressée au processus d'adaptation de maillage tridimensionnels. Ce qui représente un niveau de difficulté bien supérieurs étant donné la présence d'une métrique riemannienne.

\clsignature{William Bussière}