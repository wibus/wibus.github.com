\insertcvheader

%%LANGUAGES
\cvplaincontent{
 \textbf{Langues}~: Français, Anglais (parlé et écrit)
}

%% SKILLS
\cvsection{COMPÉTENCES}

\cvplaincontent{ 
 \textbf{Systèmes d'exploitation}~: Linux, Windows, Mac OS X\\
 \textbf{Langages}~: C++, GLSL, JavaScript, Python, Java, LaTeX, SVG\\ 
 \textbf{Librairies}~: Qt, OpenGL, CUDA, VTK, NodeJS, Electron, Unity\\ 
 \textbf{Techniques d'infographie}~: path tracing, radiosité, rendu volumique
}

%% EDUCATION
\cvsection{FORMATION SCOLAIRE}

\cventry{2015~--~2017}{Maîtrise recherche en génie informatique}{Polytechnique Montréal}{
 Adaptation de maillage en parallèle sur CPU et GPU\\
 Résultats de recherche présentés à l'USNCCM 14%\\
 %Nomination pour le meilleur mémoire de l'année 2017
}

\cventry{2010~--~2015}{Baccalauréat en génie logiciel}{Polytechnique Montréal}{
 Moyenne cumulative : 3,93/4\\
 Orientation multimédia
}

%% R&D CONTRIBUTIONS
\cvsection{CONTRIBUTIONS À LA RECHERCHE ET DÉVELOPPEMENT}

\cvplaincontent{
    \textit{\textbf{Bussière, W.}, Guibault, F. (2017) Hybrid CPU/GPU parallel unstructured mesh adaptation. 14th U.S. National Congress on Computational Mechanics. (présentation orale des travaux de maitrise)}\\
    Contributions~: Conception des tests, génération des résultats et production de figures et tableaux. La rédaction des transparents ainsi que la présentation orale ont été exécutées par mon directeur de recherche~: François Guibault.
}

%% WORK
\cvsection{EXPÉRIENCE PROFESSIONNELLE}

\cventry{Novembre 2017}{Développeur logiciel de rendu AR/VR}{Autodesk}{
    Développement d'un engin de visualisation architecturalle pour les appareils de réalité virtuelle\\
    Implémenter des lumières en boîtes pour les plateformes supportant OpenGL\\
    Optimiser l'accès aux lumières locales pour le \textit{light baking} sur GPU\\
    Page web~: \href{https://www.autodesk.com/products/stingray/overview}{autodesk.com/products/stingray/overview}\\
    Processus de développement Agile/Scrum 
}

\cventry{Été 2014}{Stage en développement logiciel}{Dental Wings}{
 Développement d'une station de numérisation dentaire intraorale\\
 Implémenter un moteur d'ombres portées à l'aide d'OpenGL/GLSL\\
 Concevoir des gestuelles Leap Motion pour la manipulation de mâchoires\\
 Page web~: \href{http://www.dentalwings.com/fr/produits/scanneur-intraoral/}{dentalwings.com/fr/produits/scanneur-intraoral}\\
 Processus de développement Agile/Scrum 
}

\cventry{2013~--~2014}{Directeur technique pour MÉTIS}{Polytechnique Montréal}{
 Partenariat avec le Centre de réadaptation Marie Enfant du CHU Sainte-Justine\\
 Conception d'un fauteuil roulant motorisé pour jeunes enfants\\
 Diriger l'équipe logicielle au sein d'un projet étudiant multidisciplinaire\\
 Communiquer l'avancement du travail au conseil d'administration hebdomadairement\\
 Évaluer et sélectionner des technologies pour la réalisation du projet\\
 Documenter des solutions choisies et des procédures d'installation
}

\cventry{2012~--~2013}{Stage en développement logiciel}{Zimmer CAS}{
 Automatiser l'assemblage d'outils chirurgicaux pour le remplacement du genou\\
 Scripter en C\# et C++ CLI les opérations géométriques exécutées par SOLIDWORKS\\
 Manipuler et afficher des IRM médicales à l'aide des librairies GDCM et VTK\\
 Valider des librairies tierces et documenter les applications développées\\
 Concevoir des interfaces graphiques avec la librairie Qt\\
 Processus de développement cascade, en transition vers Agile/Scrum
}

%\cventry{Été 2011}{Restructuration du cours INF1010}{Polytechnique Montréal}{
% Concevoir des exercices et des exemples portant sur la programmation orientée objet\\
% Rédiger des notes de cours sur la librairie STL et les interfaces graphiques\\
% Concevoir des mini-évaluations pour les élèves\\
% Restructurer le site web du cours sur \textit{Moodle}
%}


%% PROJECTS
\cvsection{PROJETS ET RÉALISATIONS}

\cventry{2015~--~2016}{Production d'un court métrage d'animation}{Projet personnel}{
 \textit{The Fruit -- A Moving Picture}~: \href{http://wibus.github.io/ExTh}{wibus.github.io/ExTh}\\
 Logiciels d'animation, rendu et postproduction développés expressément pour cette animation\\
 Le système produit des images par path tracing sur CPU de manière parallèle et distribuée\\
 Le logiciel de postproduction réduit le bruit, applique des effets d'éblouissements, fait un mappage de ton local temporellement et applique une correction gamma sur les images\\
 La scène ne contient aucun triangle; tous les objets sont le résultat d'opérations booléennes sur des surfaces analytiques quadriques (plans, sphères, paraboloïdes, hyperboloïdes)
}

\cventry{2011~--~2016}{Conception d'un moteur de simulations}{Projet personnel}{
 \textit{Experiemental Theatre} est une librairie  C++ qui regroupe les quatre modules suivants en exploitant la métaphore du théâtre pour nommer les classes et définir leurs responsabilités~:\\
 $\bullet$ \textit{Cellar Workbech}~: Structures de données, patrons de conception génériques, journalisation, primitives géométriques, gestion de ressources, classes wrapper pour OpenGL, etc.\\
 $\bullet$ \textit{Prop Room 2D}~: Framework de rendu et de physique newtonienne pour des géométries 2D\\
 $\bullet$ \textit{Prop Room 3D}~: Framework de rendu par path tracing en parallèle et distribué sur CPU\\
 $\bullet$ \textit{Scaena}~: Framework pour la création et la gestion d'entités dans une simulation ou un jeu\\
 Une dizaine de projets personnels sont basés sur cette librairie, dont mon projet de maîtrise\\
 Disponible sur Github : \href{http://www.github.com/wibus/ExperimentalTheatre}{github.com/wibus/ExperimentalTheatre}
}

\cventry{Printemps\\2015}{Développement d'un jeu vidéo pour Laval Virtual}{Projet universitaire}{
 Jeu de course à obstacles pour la Kinect 2 avec compagnons sur tablettes électroniques\\
 Les compagnons alliés déposent des objets sur le parcours pour avantager le coureur\\
 Les compagnons ennemis ajoutent des obstacles pour freiner le coureur\\
 Les tablettes des compagnons sont connectées par WiFi à l'ordinateur qui exécute le jeu\\
 La trajectoire du parcours est générée aléatoirement pour une expérience unique à chaque fois\\
 Le jeu est physiquement exigeant pour le coureur et favorise la coopération entre compagnons
}

\cventry{Printemps\\2012 et 2014}{Développement d'un jeu de hockey sur coussin d'air}{Projet universitaire}{
 Interfacer une application Java avec un framework C++ par JNI\\
 Concevoir une architecture logicielle basée sur les patrons Composite et Visiteur\\
 Appliquer les patrons de conception~: Façade, Commande, Singleton, Observateur, etc.\\
 Concevoir l'interface graphique en Swing\\
 Afficher la scène 3D à l'aide d'OpenGL\\
 Implémenter la physique de jeu en 2D
}

\cventry{Printemps\\2011}{Assemblage et programmation d'un robot}{Projet universitaire}{
 Assembler et installer les engrenages d'un petit moteur électrique\\
 Monter la carte mère pour accueillir un microcontrôleur ATMega16 et un pont en H\\
 Interfacer un capteur magnétique, un capteur infrarouge et une mémoire externe\\
 Produire des mélodies à l'aide du module PWM et d'un haut-parleur piézoélectrique\\
 Développer un programme C++ pour la traversée d'un parcours à obstacles
}


%% SCOLARSHIPS
\cvsection{BOURSES ET PRIX}
 
\cventry{Hiver 2015}{Bourse du CRSNG}{}{
 Bourse d'entrée à la maîtrise recherche
}

\cventry{Hiver 2015}{Bourse du FRQNT}{}{
 Bourse d'entrée à la maîtrise recherche
}

\cventry{Hiver 2015}{Bourse de Polytechnique}{}{
 Bourse d'entrée à la maîtrise pour l'excellence du dossier académique
}

\cventry{Hiver 2011}{Prix Philip et Lily Malouf}{}{
 Équipe gagnante du projet intégrateur de première année
}


%% REFERENCES
\cvsection{RÉFÉRENCES}

\cvplaincontent{
    Olivier Dionne, \textit{Chef d'équipe chez Autodesk}~: olivier.dionne@autodesk.com \\
	François Guibault, \textit{Directeur de recherche}~: francois.guibault@polymtl.ca \\
	Julien Marbach, \textit{Chef d'équipe chez Dental Wings}~: julien.marbach@dental-wings.com\\
	Eric Brosseau, \textit{Chef d'équipe chez Zimmer CAS}~: brosse@sympatico.ca
}